%DRÁBIK, Peter: \emph{Využitie MS HoloLens 2 vo vzdelávaní}. [Diplomová práca]. {--} Paneurópska vysoká škola. Fakulta informatiky. {--} Školiteľ
Táto práca skúma potenciál využitia zmiešanej reality (MR) vo vzdelávaní, konkrétne v kontexte výučby biológie na stredných školách so zameraním na nervovú sústavu. 
Cieľom bolo vytvoriť a otestovať výukovú aplikáciu založenú na technológii MR, ktorá by umožnila lepšie sprístupniť zvolenú tému vzdelávania.
Teoretická časť práce popisuje problematiku umelej reality a potenciál jej využitia v oblasti vyučovania biológie.
Praktická časť práce sa zaoberá procesom tvorby aplikácie, ktorá bola následne podrobená pedagogickému výskumu v školských podmienkách.
Výsledky kvalitatívneho výskumu naznačujú pozitívny postoj študentov k testovanej aplikácii a absenciu istých miskoncepcií v experimentálnej skupine. 
Avšak, kvantitatívny výskum poukazuje na potrebu ďalšieho pedagogického výskumu na väčšej vzorke respondentov, aby sa potvrdili zistenia a eliminovali nejednoznačnosti vo výsledkoch.
