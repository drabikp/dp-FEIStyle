Cieľom práce bolo demonštrovať potenciál využitia zmiešanej reality vo vyučovaní biológie na stredných školách prostredníctvom zariadenia MS HoloLens 2. Zamerali sme sa na tematiku nervovej sústavy, ktorá sa preukázala ako 
oblasť, pri výuke ktorej môžu často vznikať miskoncepcie.

Pre aplikáciu, ktorá je súčasťou praktickú časti tejto práce, sme vytvorili trojrozmerný model nervovej bunky. Ten sme následne ďalej využili v aplikácií, ktorej účelom bolo sprístupniť študentom princíp prenosu nervového
vzruchu. 
Aplikáciu sme vyvinuli pomocou Unreal Engine a následne sme ju podrobili pedagogickému výskumu v školských podmienkách. 

Výsledky kvalitatívneho výskumu naznačujú pozitívny postoj študentov k testovanej aplikácii.
Kvantitatívny výskum však poukazuje na potrebu ďalšieho pedagogického výskumu na väčšej vzorke respondentov, aby sa potvrdili zistenia a eliminovali nejednoznačnosti vo výsledkoch.

Na základe výsledkov výskumu konštatujeme, že hoci nami vytvorenú aplikáciu by bolo potrebné podrobiť ďalšiemu výskumu a prípadne odstrániť nedostatky, ktoré by takýmto výskumom boli zistené, 
spätná väzba respondentov indikuje, že téma sprístupnená takýmto spôsobom je pre nich atraktívnejšia. Tým pokladáme cieľ práce za splnený.


