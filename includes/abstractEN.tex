This thesis explores the potential of using Mixed Reality (MR) in education, specifically in the context of teaching biology in secondary schools with a focus on the nervous system. 
The aim was to develop and test an educational application based on MR technology that would allow better accessibility to the chosen educational topic.
The theoretical part of the thesis describes the subject of artificial reality and the potential of its use in the field of biology teaching.
The practical part of the thesis deals with the process of creating the application, which was subsequently subjected to pedagogical research in school conditions.
The results of the qualitative research indicate a positive attitude of students towards the tested application and the absence of certain misconceptions in the experimental group. 
However, the quantitative research points to the need for further pedagogical research on a larger sample of respondents in order to confirm the findings and eliminate ambiguities in the results.