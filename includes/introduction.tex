Technologické možnosti 21. storočia do našich životov prirodzene prinášajú príležitosti zdokonaliť zaužívané spôsoby vykonávania mnohých činností. Jednou z oblastí, ktoré nielenže poskytujú priestor na využitie moderných technológií,
ale sa postupom času bez týchto výdobytkov doby ani nezaobídu, je práve vzdelávanie. Téma tejto práce reflektuje súčasné trendy {--} zaoberá sa možnosťou využitia headsetu pre zmiešanú realitu HoloLens 2 od spoločnosti Microsoft vo 
vzdelávaní prírodovedných predmetov. 

V tejto práci uvedieme čitateľa do problematiky virtuálnej, rozšírenej a zmiešanej reality a popíšeme niektoré možnosti vývoja aplikácií pre HoloLens. Následne identifikujeme možnosť použitia zmiešanej reality v oblasti 
vyučovania biológie na stredných školách a zameriame sa na lepšie sprístupnenie vybranej témy biologického vzdelávania (nervová sústava) a elimináciu miskoncepcií, ktoré môžu pri sprístupňovaní tejto témy vznikať.

Praktickú časť práce tvorí aplikácia pre zmiešanú realitu, ktorá bola navrhnutá na základe rešerše odbornej biologickej a didaktickej literatúry v teoretickej časti práce. 
Túto aplikáciu následne podrobíme pedagogickému výskumu v školských podmienkach.

Výsledky výskumu sú vyhodnotené štatistickými metódami a primerane interpretované. 

Cieľom práce je demonštrovať potenciál využitia zmiešanej reality vo vyučovaní biológie. Veríme, že sa nám tento cieľ podarilo naplniť.
